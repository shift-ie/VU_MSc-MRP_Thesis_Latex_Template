\chapter{Introduction}\label{s:intro} \FloatBarrier

Human movement is a complex and multi-faceted phenomenon that has been studied for decades. It encompasses a wide range of activities, from simple everyday movements such as walking and reaching, to more complex movements such as athletic performance and dance \cite{pereira2017effect,tyler2016new}. The study of human movement is an interdisciplinary field that draws on a range of scientific disciplines, including biomechanics, physiology, psychology, and neuroscience \cite{boyd2017effects}. Understanding how the body moves, how it is controlled and how it interacts with the environment is important for a wide range of applications, including sport and exercise performance, rehabilitation, and the prevention of injury\cite{van2016neuroplasticity}.

% NOTE: for explanation how to use please check MacroFile1.text
\figuremacro{example1}{ExampleLabel}{Example Title}{Example figure description}

One of the key challenges in studying human movement is the variability and complexity of the movements themselves\cite{miller2017predictors,muller2018physical,pereira2017effect}. There is a great deal of individual variation in the way people move, which can make it difficult to generalize findings across different populations. In addition, human movement is influenced by a range of factors, including physiological, psychological and environmental factors \cite{decre2016reliability,johnson2018influence}. As a result, understanding the underlying mechanisms of human movement requires a multi-disciplinary approach that incorporates a range of different research methodologies.

The aim of this thesis is to explore the factors that influence human movement and to investigate the underlying mechanisms that drive it. Specifically, the thesis will focus on the study of human movement in relation to physical activity and exercise performance. Physical activity is a key determinant of health, and understanding the mechanisms by which it affects the body can have important implications for the prevention and treatment of a range of health conditions \cite{lee2017exploring,maclellan2017quantifying,boyd2017effects}.

\figuremacroG{multi1}{multi2}{GridExampleLabel}{Grid Example Title}{Example grid figure description, with reference to Figure \ref{fig:ExampleLabel}}

The thesis will be structured into four chapters, each of which will focus on a different aspect of human movement. Chapter one will provide an overview of the key concepts and theoretical frameworks that underpin the study of human movement, including the role of biomechanics, physiology, and psychology. Chapter two will focus on the measurement of physical activity and exercise performance, including the use of wearable technology and other objective measures. Chapter three will explore the physiological and psychological mechanisms that underpin the effects of physical activity on the body, including the role of exercise intensity, duration, frequency and a random mention of \nomenclature{DMSO}. Finally, chapter four will investigate the role of environmental factors in influencing human movement, including the impact of temperature, humidity, and altitude.

% nice table style with optional footnote. Avoid vertical boundaries and bold face text. 
\begin{table}[htp]
	\centering \ra{1.2}
	\caption[Example Title]{Example table description}
	\begin{tabular}{@{}lll@{}} 
		\toprule
									& Fancy Symbols				& Some Numbers \\ 
		\midrule
		Row							& $m$						& 42\\
		Another row					& $\oint^a$ 				& 360 \\
		Row 4						& ${\zeta_z}$				& $e$ \\
		\bottomrule
		\multicolumn{3}{l}{\footnotesize{$^a$Some symbols require the Amsmath package}}
	\end{tabular}
	\label{tab:example}
\end{table}

In order to achieve these aims, a range of different research methodologies will be employed. These will include laboratory-based studies, field studies, and epidemiological studies. Laboratory-based studies will be used to investigate the physiological and psychological mechanisms that underpin the effects of physical activity on the body. Field studies will be used to explore the impact of environmental factors on human movement, while epidemiological studies will be used to investigate the relationship between physical activity and health outcomes.

Overall, this thesis aims to contribute to the growing body of knowledge on human movement and its underlying mechanisms. By exploring the factors that influence human movement in relation to physical activity and exercise performance, the thesis will provide new insights into how the body responds to different types of physical activity, and how this can be optimized for health and performance. The findings of this thesis may have important implications for the development of interventions aimed at promoting physical activity, improving exercise performance, and preventing and treating a range of health conditions.

